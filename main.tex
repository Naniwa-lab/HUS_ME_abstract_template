\documentclass{HUS_abstract}

%====================================================================
% ここから上は編集しない
%====================================================================

%--------------------------------------------------------------------
% 基本情報(必ず記入すること)
%--------------------------------------------------------------------
\titleja{Ta$_2$Co$_7$の結晶構造及び圧縮変形挙動調査}
\titleen{Crystal Structure and Compressive Deformation Behavior of Ta$_2$Co$_7$}

\studentid{1223999}
\authorja{北科大 太郎}
\advisor{波須 機械 教授}

\authoren{Taroh KIKAI, Sakura GIJYUTSU}

\keywords{mille-feuille structure, Co$_3$Ta, deformation behavior, Co-Ta binary system}

%====================================================================
\begin{document}
\maketitle
%====================================================================

%--------------------------------------------------------------------
\section{緒言}
%--------------------------------------------------------------------
ここに緒言を書く.
研究背景,本研究の目的,関連研究との位置付けなどを簡潔に述べること\cite{ref1}.

%--------------------------------------------------------------------
\section{実験方法}
%--------------------------------------------------------------------
\subsection{X線回折法による結晶構造解析}
試料作製方法,測定条件,使用した装置などを簡潔に記述する.

\subsection{圧縮試験による変形挙動評価}
圧縮試験片の寸法,試験条件,評価方法について述べる.

%--------------------------------------------------------------------
\section{実験結果および考察}
%--------------------------------------------------------------------
得られた結果を図表とともに示し,考察を行う.

%--------------------------------------------------------------------
\section{結言}
%--------------------------------------------------------------------
本研究で得られた結論を簡潔にまとめる.

%--------------------------------------------------------------------
\section*{今後の課題}
%--------------------------------------------------------------------
今後の研究課題や展望について述べる.

\begin{figure}[b]
  \centering
  \includegraphics[width=0.9\linewidth]{./figure/1.eps}
  \caption{図のキャプションを書く}
  \label{fig:sample}
\end{figure}


%--------------------------------------------------------------------
\section*{謝辞}
%--------------------------------------------------------------------
指導教員や研究に協力した方々への謝辞を書く.

%--------------------------------------------------------------------
\begin{thebibliography}{99}

%--------------------------------------------------------------------
\bibitem{ref1}
著者名:論文タイトル,雑誌名,巻,ページ(年).

\bibitem{ref2}
Author, A. B., Title of the paper, Journal name, Vol., pp. (Year).
\end{thebibliography}



%====================================================================
\end{document}
%====================================================================