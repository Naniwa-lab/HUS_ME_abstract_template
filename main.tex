\documentclass{HUS_abstract}

%====================================================================
% ここから上は編集しない
%====================================================================

%--------------------------------------------------------------------
% 基本情報(必ず記入すること)
%--------------------------------------------------------------------
\titleja{ゾウの卵の発見とその力学的特性に関する研究}
\titleen{Discovery of Elephant Eggs and Investigation of Their Mechanical Properties}

\studentid{12XXXXX}
\authorja{北科大 太郎}
\advisor{指導 教員 教授}

\authoren{Taro HOKKAIDO}

\keywords{elephant egg, fictional object, mechanical property, imaginary experiment}

%====================================================================
\begin{document}
\maketitle
%====================================================================

%--------------------------------------------------------------------
\section{緒言}
%--------------------------------------------------------------------
本研究は,「ゾウの卵」という架空の対象を題材として,卒業論文梗概の記述例を示すことを目的とする.
ゾウは哺乳類であり卵を産まないことが広く知られているが,あえて非現実的な対象を設定することで,特定分野への依存を避けた一般的な記述例を提供する.
本研究の位置付けについては既存の架空研究を参考にした\cite{ref1}.

%--------------------------------------------------------------------
\section{手法}
%--------------------------------------------------------------------
ゾウの卵は仮想的に発見されたものと仮定し,その直径,質量,および殻の硬さを測定した.
測定方法や条件はすべて仮定に基づくものであり,本節では論文記述の形式のみを示すことを目的とする.

%--------------------------------------------------------------------
\section{結果および考察}
%--------------------------------------------------------------------
測定の結果,ゾウの卵は非常に大きく,かつ高い耐荷重性を有するという結果が得られた.
これらの結果は現実の生物学的知見とは一致しないが,結果と考察を分けて記述する際の構成例として有用である.

\begin{figure}[b]
  \centering
  \includegraphics[width=0.9\linewidth]{./figure/1.eps}
  \caption{ゾウの卵(想像図)}
  \label{fig:elephant_egg}
\end{figure}

%--------------------------------------------------------------------
\section{結言}
%--------------------------------------------------------------------
本研究では,架空の対象であるゾウの卵を用いて,卒業論文梗概の一般的な記述例を示した.
本テンプレートを用いることで,分野に依存しない安全な例文を提供できることを確認した.

%--------------------------------------------------------------------
\section*{今後の課題}
%--------------------------------------------------------------------
今後は,各自の研究分野に応じて,本テンプレート内の文章を適切な内容に置き換えることが課題である.

%--------------------------------------------------------------------
\section*{謝辞}
%--------------------------------------------------------------------
本研究は実在しないため,謝辞も形式例としてのみ記載する.

%--------------------------------------------------------------------
\begin{thebibliography}{99}

\bibitem{ref1}
架空 太郎:ゾウの卵概論,空想科学ジャーナル,1,pp.1--10(20XX).

\end{thebibliography}

%====================================================================
\end{document}
%====================================================================